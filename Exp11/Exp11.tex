%!Mode:: "TeX:UTF-8"
\documentclass[a4paper,11pt,UTF8]{ctexart}

\usepackage{indentfirst} %缩进
\usepackage{xeCJK}    %使用系统字体
\usepackage{fancyhdr} %自定义页眉页脚
\pagestyle{empty}                   %不设置页眉页脚
\usepackage{amsmath, amsthm, amssymb, amsfonts} %数学公式
\usepackage[a4paper,left=3cm,right=3cm,top=3cm,bottom=3cm]{geometry}
%\usepackage[tmargin=1in,bmargin=1in,lmargin=1.25in,rmargin=1.25in]{geometry}.
\usepackage{booktabs} %插入表格
\usepackage[section]{placeins} %避免浮动
\usepackage{listings} %插入代码
\usepackage{ctex}     %中文宏包
\usepackage[svgnames, table]{xcolor} %彩色表格
\usepackage{algorithm}          %伪代码
\usepackage{algorithmicx}
\usepackage{algpseudocode}
\usepackage{algorithm,algpseudocode,float}
\usepackage{lipsum}
\usepackage{enumitem}           %调整列举环境
\usepackage{url}
\usepackage{fontspec,xunicode}
\defaultfontfeatures{Mapping=tex-text} %如果没有它,会有一些 tex 特殊字符无法正常使用,比如连字符。

\usepackage{graphicx}
\graphicspath{{imgs/}}

%%%%%%%%%%%%%%%%%%%%%%%%%%%%%%%%%%%%%%%%%%%%%%%%%%%%%%%%%%%%%%%%
% 缩进及行间距
%%%%%%%%%%%%%%%%%%%%%%%%%%%%%%%%%%%%%%%%%%%%%%%%%%%%%%%%%%%%%%%%
\setlength{\parindent}{22pt} %重新定义缩进长度
\setlength{\baselineskip}{20pt}  %定义行间距
%\renewcommand{\baselinestretch}{1.1} %定义行间距

%%%%%%%%%%%%%%%%%%%%%%%%%%%%%%%%%%%%%%%%%%%%%%%%%%%%%%%%%%%%%%%%
% 列表设置
%%%%%%%%%%%%%%%%%%%%%%%%%%%%%%%%%%%%%%%%%%%%%%%%%%%%%%%%%%%%%%%%
\setenumerate{fullwidth,itemindent=\parindent,listparindent=\parindent,itemsep=0ex,partopsep=0pt,parsep=0ex}
\setenumerate[2]{label=\alph*),leftmargin=1.5em}  %二级item设置
\setitemize{itemindent=38pt,leftmargin=0pt,itemsep=-0.4ex,listparindent=26pt,partopsep=0pt,parsep=0.5ex,topsep=-0.25ex}
\setdescription{itemindent=38pt,leftmargin=0pt,itemsep=-0.4ex,listparindent=26pt,partopsep=0pt,parsep=0.5ex,topsep=-0.25ex}

%%%%%%%%%%%%%%%%%%%%%%%%%%%%%%%%%%%%%%%%%%%%%%%%%%%%%%%%%%%%%%%%
% 图的标题行间距设置
%%%%%%%%%%%%%%%%%%%%%%%%%%%%%%%%%%%%%%%%%%%%%%%%%%%%%%%%%%%%%%%%
\newcommand{\bottomcaption}{%
\setlength{\abovecaptionskip}{6pt}%
\setlength{\belowcaptionskip}{6pt}%
\caption}


%%%%%%%%%%%%%%%%%%%%%%%%%%%%%%%%%%%%%%%%%%%%%%%%%%%%%%%%%%%%%%%%
% 字体定义
%%%%%%%%%%%%%%%%%%%%%%%%%%%%%%%%%%%%%%%%%%%%%%%%%%%%%%%%%%%%%%%%
\setmainfont{Times New Roman}  %默认英文字体.serif是有衬线字体sans serif无衬线字体
\setmonofont{Consolas}
\setCJKmainfont[ItalicFont={楷体}, BoldFont={黑体}]{宋体}%衬线字体 缺省中文字体为
\setCJKsansfont{黑体}
\punctstyle{hangmobanjiao}
%-----------------------xeCJK下设置中文字体------------------------------%
\setCJKfamilyfont{song}{SimSun}                             %宋体 song
\newcommand{\song}{\CJKfamily{song}}
\setCJKfamilyfont{fs}{FangSong}                      %仿宋  fs
\newcommand{\fs}{\CJKfamily{fs}}
\setCJKfamilyfont{ktgb}{KaiTi}                      %楷体2312 ktgb
\newcommand{\ktgb}{\CJKfamily{ktgb}}
\setCJKfamilyfont{yh}{Microsoft YaHei}                    %微软雅黑 yh
\newcommand{\yh}{\CJKfamily{yh}}
\setCJKfamilyfont{hei}{SimHei}                              %黑体  hei
\newcommand{\hei}{\CJKfamily{hei}}
\setCJKfamilyfont{hwxk}{STXingkai}                                %华文行楷  hwxk
\newcommand{\hwxk}{\CJKfamily{hwxk}}
%------------------------------设置字体大小------------------------%
\newcommand{\shiyanbaogao}{\fontsize{36pt}{\baselineskip}\selectfont}
\newcommand{\chuhao}{\fontsize{42pt}{\baselineskip}\selectfont}     %初号
\newcommand{\xiaochuhao}{\fontsize{36pt}{\baselineskip}\selectfont} %小初号
\newcommand{\yihao}{\fontsize{28pt}{\baselineskip}\selectfont}      %一号
\newcommand{\erhao}{\fontsize{21pt}{\baselineskip}\selectfont}      %二号
\newcommand{\xiaoerhao}{\fontsize{18pt}{\baselineskip}\selectfont}  %小二号
\newcommand{\sanhao}{\fontsize{15.75pt}{\baselineskip}\selectfont}  %三号
\newcommand{\sihao}{\fontsize{14pt}{\baselineskip}\selectfont}       %四号
\newcommand{\xiaosihao}{\fontsize{12pt}{\baselineskip}\selectfont}  %小四号
\newcommand{\wuhao}{\fontsize{10.5pt}{\baselineskip}\selectfont}    %五号
\newcommand{\xiaowuhao}{\fontsize{9pt}{\baselineskip}\selectfont}   %小五号
\newcommand{\liuhao}{\fontsize{7.875pt}{\baselineskip}\selectfont}  %六号
\newcommand{\qihao}{\fontsize{5.25pt}{\baselineskip}\selectfont}    %七号

%%%%%%%%%%%%%%%%%%%%%%%%%%%%%%%%%%%%%%%%%%%%%%%%%%%%%%%%%%%%%%%%
% 图题字体大小相同
%%%%%%%%%%%%%%%%%%%%%%%%%%%%%%%%%%%%%%%%%%%%%%%%%%%%%%%%%%%%%%%%
\usepackage{caption}
\captionsetup{font={footnotesize}}   % footnotesize = 9pt
\captionsetup[lstlisting]{font={footnotesize}}

%%%%%%%%%%%%%%%%%%%%%%%%%%%%%%%%%%%%%%%%%%%%%%%%%%%%%%%%%%%%%%%%
% 重定义枚举编号为 1),2)...
%%%%%%%%%%%%%%%%%%%%%%%%%%%%%%%%%%%%%%%%%%%%%%%%%%%%%%%%%%%%%%%%
\renewcommand{\labelenumi}{\theenumi)}

%%%%%%%%%%%%%%%%%%%%%%%%%%%%%%%%%%%%%%%%%%%%%%%%%%%%%%%%%%%%%%%%
% 标题名称中文化
%%%%%%%%%%%%%%%%%%%%%%%%%%%%%%%%%%%%%%%%%%%%%%%%%%%%%%%%%%%%%%%%
\renewcommand\figurename{\hei 图}
\renewcommand\tablename{\hei 表}
\renewcommand\lstlistingname{\hei 代码}
\renewcommand{\algorithmicrequire}{\textbf{输入:}}
\renewcommand{\algorithmicensure}{\textbf{输出:}}
\newtheorem{define}{定义}

%%%%%%%%%%%%%%%%%%%%%%%%%%%%%%%%%%%%%%%%%%%%%%%%%%%%%%%%%%%%%%%%
% 代码设置
%%%%%%%%%%%%%%%%%%%%%%%%%%%%%%%%%%%%%%%%%%%%%%%%%%%%%%%%%%%%%%%%
\lstset{
 columns=fixed,
 numbers=left,                                        % 在左侧显示行号
 numberstyle=\tiny\color{gray},                       % 设定行号格式
 frame=single,                                        % 单线背景边框
 breaklines=true,                                     % 设定LaTeX对过长的代码行进行自动换行
 keywordstyle=\color[RGB]{40,40,255},                 % 设定关键字颜色
 numberstyle=\footnotesize\color{darkgray},
 commentstyle=\it\color[RGB]{0,96,96},                % 设置代码注释的格式
 stringstyle=\rmfamily\slshape\color[RGB]{128,0,0},   % 设置字符串格式
 showstringspaces=false,                              % 不显示字符串中的空格
 language=java,                                        % 设置语言
 basicstyle=\linespread{1.0}\xiaowuhao\ttfamily,                      % 字体字号
 %lineskip=10pt,
 %baselinestretch=1,
}

%%%%%%%%%%%%%%%%%%%%%%%%%%%%%%%%%%%%%%%%%%%%%%%%%%%%%%%%%%%%%%%%
% 伪代码分页
%%%%%%%%%%%%%%%%%%%%%%%%%%%%%%%%%%%%%%%%%%%%%%%%%%%%%%%%%%%%%%%%
\makeatletter
\renewcommand{\ALG@name}{算法}
\newenvironment{breakablealgorithm}
  {% \begin{breakablealgorithm}
   \begin{center}
     \refstepcounter{algorithm}% New algorithm
     \hrule height.8pt depth0pt \kern2pt% \@fs@pre for \@fs@ruled
     \renewcommand{\caption}[2][\relax]{% Make a new \caption
       {\raggedright\textbf{\ALG@name~\thealgorithm} ##2\par}%
       \ifx\relax##1\relax % #1 is \relax
         \addcontentsline{loa}{algorithm}{\protect\numberline{\thealgorithm}##2}%
       \else % #1 is not \relax
         \addcontentsline{loa}{algorithm}{\protect\numberline{\thealgorithm}##1}%
       \fi
       \kern2pt\hrule\kern2pt
     }
  }{% \end{breakablealgorithm}
     \kern2pt\hrule\relax% \@fs@post for \@fs@ruled
   \end{center}
  }
\makeatother

% =============================================
% Part 1 Edit the info
% =============================================

\newcommand{\major}{物理学院}
\newcommand{\name}{黄阅迅,李秋阳}
\newcommand{\stuid}{PB18020631,PB18020567}
\newcommand{\group}{20}
\newcommand{\newdate}{\today}


\newcommand{\course}{电子线路实验(1)}
\newcommand{\newtitle}{数据选择器}

% =============================================
% Part 1 Main document
% =============================================
\begin{document}
\thispagestyle{empty}
\begin{figure}[h]
  \begin{minipage}{0.6\linewidth}
    \centerline{\includegraphics[width=\linewidth]{logo.png}}
  \end{minipage}
  \hfill
  \begin{minipage}{.4\linewidth}
    \raggedleft
    \begin{tabular*}{.8\linewidth}{ll}
      学院: & \underline\major   \\
      姓名: & \underline\name    \\
      学号: & \underline\stuid   \\
      组号:  & \underline\group   \\
      日期: & \underline\newdate \\
    \end{tabular*}
  \end{minipage}
\end{figure}

\begin{table}[!htbp]
  \centering
  \begin{tabular*}{\linewidth}{llllll}
    课程名称:  \underline\course   \qquad\qquad 实验题目:  \underline\newtitle  
  \end{tabular*}
\end{table}

% =============================================
% Part 2 Main document
% =============================================

\section{实验目的}

请参看预习报告。

\section{实验原理}

请参看预习报告。

\section{实验内容与步骤}
\subsection{实验内容}
	blablabla
\subsection{实验步骤}
	blablabla

\section{实验数据处理与分析}
\subsection{实验内容1}
	blablabla
\subsection{误差分析1}
	blablabla

\section{实验总结}
blablabla
\section{实验思考题}
\subsection{问题一}
1.试用4选1数据选择器实现一个简单的交通灯故障检测电路。
(要求:每一组信号灯由红、黄、绿三盏灯组成。正常工作情况下,
任何时刻只有一盏灯点亮,而且只允许有一盏灯点亮。而当出现其他五
种点亮状态时,认为电路发生故障,这时要求发出故障信号,以提醒维护人员去修理。)

答:首先指派每个灯的逻辑符号:

红灯亮---A; 黄灯亮---B;绿灯亮---C;

则可以写出逻辑表达式:
\begin{equation}
  Y=ABC+A'BC+AB'C+ABC'+A'B'C'
\end{equation}
即三个灯中有且仅有一盏灯亮时正常,否则发出故障信号。

再指派:
\begin{equation}
  A---A_1;\;B---A_0
\end{equation}
上述逻辑表达式可以写成最小项形式:
\begin{equation}
  Y=m_31+m_1C+m_2C'+m_0C'
\end{equation}
对比数据选择器的输出表达式:$Y=\sum_0^3m_iD_i$。
则可得$D_i$的指派:
\begin{equation}
  D_0---C';D_1---C;D_2---C';D_3---1;
\end{equation}
真值表如表 \ref{tab:T1}所示。
\begin{table}[H]\label{tab:T1}
  \centering
  \begin{tabular}{|ccc|c|}\hline
   $A$&$B$&$C$ &$Y$
   \\\hline
   0&0&0 &1
   \\\hline
   0&0&1 &0
   \\\hline
   0&1&0 &0
   \\\hline
   0&1&1 &1
   \\\hline
   1&0&0 &0
   \\\hline
   1&0&1 &1
   \\\hline
   1&1&0 &1
   \\\hline
   1&1&1 &1
   \\\hline
  \end{tabular}
  \caption{交通灯检测器真值表}
 \end{table}
则由上述分析容易得到电路如\ref {fig:T1}所示。

\begin{figure}[htbp]
  \centering
  \fbox{\includegraphics[width=0.8\linewidth]{T1.png}}
  \caption{思考题一电路图}
  \label{fig:T1}
  \end{figure}
\subsection{问题二}
2.数据选择器和译码器都可以实现逻辑函数,试描述他们的特点以及不同之处。

答:译码器将每个二进制代码赋予的特定含义“翻译”过来,转换成相应的信息符号(输出信号)。
其输出特点是将每一个最小项的反输出为其值的反,意即:$Y'_i=m_i'$。

而数据选择器则是根据输入的地址,选择对应地址的信息输出。其输出特点是将每个最小项与其对应的
地址相乘后在求和所有的项,意即:$Y=\sum m_iD_i$。

其最大的不同是数据选择器只有一个输出,而译码器有多个输出端,因此在实现特定逻辑函数时,后者往往需要一个与操作。
而译码器实现逻辑函数非常直接,将逻辑函数表示成多个最小项的非的与/或,然后在输出时将出现的项根据表达式进行与非/或非即可。

而数据选择器有多种实现方式:

一种是存储式实现,将每个最小项对应的值存储在$D_i$中,然后根据输出的最小项表达式将需要的输出值通过或门即可。这种实现方式消耗资源比较大,但是线路和原理很简单。

另一种将逻辑函数表示成最小项的或形式,然后将最后一个逻辑变量$A_n$用$D_i$实现,对比数据选择器的输出函数,将$D_i$指派为$A_n,A_n',0,1$这四种类型,此时$Y$直接对应于输出,
相当于将或门集成在了芯片当中,但需要额外的非门,这个实现较为复杂,但比较节省资源。


\begin{appendix}

\section{代码示例}

\begin{lstlisting}[caption={一段C代码},captionpos=b]
#include <stdio.h>
int main (int argc, char *argv[]){
  printf("Hello world!");
}
\end{lstlisting}

\section{表格示例}
表 \ref{tab:tab1}与表 \ref{tab:tab2}展示了表格示例
\begin{table}[!h!tbp]
\caption{一个简单的表格}\label{tab:tab1}
  \centering
  \begin{tabular}{|l|c|c|}
	\hline
	功能          &WEB         &APP         \\ \hline
	注册          &$\surd$     &$\surd$     \\ \hline
	登录          &$\surd$     &$\surd$     \\ \hline
	推送          &$\times$    &$\surd$     \\ \hline
\end{tabular}
\end{table}

\begin{table}[!h!tbp]
\caption{自定义表格}\label{tab:tab2}
  \centering
\begin{tabular*}{0.75\textwidth}{@{\extracolsep{\fill}}lcc}
    \toprule
    功能          &WEB         &APP         \\
    \midrule
    注册          &$\surd$     &$\surd$     \\
    登录          &$\surd$     &$\surd$     \\
    推送          &$\times$    &$\surd$     \\
    \bottomrule
\end{tabular*}
\end{table}


\section{图片示例}
图 \ref{fig:logo}展示了一个图片示例。
\begin{figure}[htbp]
\centering
\fbox{\includegraphics[width=0.5\linewidth]{logo}}
\caption{blablabla}
\label{fig:logo}
\end{figure}

\section{公式示例}
式 (\ref{eqa:01})展示了一个公式的例子。
\begin{equation}
S_n = \frac{X_1 + X_2 + \cdots + X_n}{n}
      = \frac{1}{n}\sum_{i}^{n} X_i
\label{eqa:01}
\end{equation}




\end{appendix}

\end{document}
